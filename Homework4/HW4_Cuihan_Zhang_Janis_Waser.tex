\documentclass[12pt]{extarticle}
\usepackage{tikz}
\usepackage{amsmath,amssymb}
\usepackage {graphicx}
\usepackage{multicol}
\usepackage{hyperref}
\usepackage{geometry}
\usepackage{watermark}
\usepackage{varwidth}
\usepackage{arydshln}
\usepackage[dvipsnames]{xcolor}
\usepackage{tcolorbox}
%\usepackage{scrlayer-scrpage}
\usepackage{forest}
\usepackage{fancyhdr}
\pagecolor{white}
 \geometry{bottom= 30mm}
\usepackage{enumitem}


 \newcommand{\assignnumber}{\color{red}number\color{black}} %tell latex number of Homework
 
 
\title{Data Management Homework 4}
\author{Cuihan Zhang \& Janis Waser} 
\fancyhead[L]{Data Management}
\fancyhead[C]{Homework 4}
\fancyhead[R]{Cuihan Zhang \& Janis Waser}
\renewcommand\headrulewidth{0pt}
\pagestyle{fancy}
\pagenumbering{arabic}

\begin{document}

\maketitle \vspace{-10mm}
\rule{\linewidth}{0.4pt}


\begin{flushleft}
\begin{enumerate}

\item

\begin{enumerate}
\item Following closures are computed:


\begin{enumerate}

\item  $(BC)^+=BCD$
\item $(BDEFG)^+=BCDEFG$
\item $(HEFG)^+=BCDEFGH$
\item $(EFG)^+=DEFG$
\item $(EFGH)^+=BCDEFGH$
\end{enumerate}
\item I is not in any functional dependency, it must be part of the key. H is only on the left side, therefore it must be part of the key, while D is only on the right side, hence it does not appear in any key. 

Now, we have to add attributes to HI to get keys while ignoring D:
HICEF, HICEG, HICFG we remove because C is in the closure of EFG which we can always get if we have at least two attributes of EFG. 

The remaining keys are: HIEF, HIEG, HIFG. They are all fit the requirements.
\end{enumerate}



%exc2
\item
\begin{enumerate}
\item Following Five FDs are computed:
\begin{enumerate}
\item $Us \rightarrow YeBa$
\item $Ye \rightarrow ReMo$
\item $UsRe \rightarrow Ba$
\item $Da \rightarrow Ye$
\item $Us \rightarrow Mo$
\end{enumerate}
\item 
Combine 1+5:
\begin{enumerate}
\item $Us \rightarrow YeBaMo$
\item $Ye \rightarrow ReMo$
\item $UsRe \rightarrow Ba$
\item $Da \rightarrow Ye$
\end{enumerate}
Now, 1+3 LHS:
\begin{enumerate}
\item $Us \rightarrow YeBaMo$
\item $Ye \rightarrow ReMo$
\item $Da \rightarrow Ye$
\end{enumerate}

Now, 1+2 RHS
\begin{enumerate}
\item $Us \rightarrow YeBa$
\item $Ye \rightarrow ReMo$
\item $Da \rightarrow Ye$
\end{enumerate}
We cannot perform any further steps, so this is a minimal cover.

 \item Towards 3NF:
 \begin{enumerate} \item 
We have three tables BaUsYe, MoReYe, DaYe.
 \item We cannot remove a table without loosing information as not relation is a subset of another.
 \item We can get a global key by analzing the FDs, Us and Da only appear on the left, while Mo and Ba only appear on the right. We conclude Us and Da are part of the key while Mo and Ba certainly are not. Ye and Re are in the closure of UsDa, hence we make \textbf{UsDa} the global key. 
 
 Since no relation contains the entire key, we have to add this relation. We end up with four tables: \textbf{BaUsYe, MoReYe, DaYe, DaUs}.
\end{enumerate}
\end{enumerate}



%exc3
\item \begin{enumerate} 
\item A and B only appear on the left and must therefore be part of the key. D on the other hand appears only on the right side, hence it is not part of the key. We add from CEF, but from AB F is already in the closure, hence we add C or E. \textbf{ABC} is key(closure is full), \textbf{ABE} as well. 
\item \begin{enumerate}
\item $CF\rightarrow E$ is a into key dependency. 
\item $AE\rightarrow C$ is a partial dependency.
\item $F \rightarrow D$ is a transitive dependency.
\item $E\rightarrow F $ is  a partial dependency.
\item $B \rightarrow F$ is a partial dependency.
\end{enumerate}
\item We can apply the algorithm to check, if we change something, then the given FDs were not in minmal cover else they are: 

There are no trivial functional dependencies. There are no union simplification(no duplicates on the left hand side), on the left side we also do not have a situation to simplify, as no left hand side who is subset has the same right side. Finally, no right side simplification is possible as the right side is already elementary.
We have not changed the FDs therefore the given FDs are a minimal cover.

\end{enumerate}





%exc4
\item 
%exc5
\item
The relation is in 1NF, because there is a fixed number of columns and no repeating group. There is no partial dependency as CE is the full key and B is not in the key. We are at least in  2NF. Now, the relation $B\rightarrow D$ is a transitive dependency, which is not allowed in 3NF, this means we are not in 3NF.
\end{enumerate}
\end{flushleft}
\end {document}